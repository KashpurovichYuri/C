\documentclass[a4paper,12pt]{article}	% тип документа

\usepackage[a4paper,top=1.3cm,bottom=2cm,left=1.5cm,right=1.5cm,marginparwidth=0.75cm]{geometry} % настройка полей

\usepackage[T2A]{fontenc}		% кодировка
\usepackage[utf8]{inputenc}		% кодировка исходного текста
\usepackage[english,russian]{babel}	% локализация и переносы
\usepackage{indentfirst}

%Рисунки
\usepackage{graphicx}

\usepackage{wrapfig}

\usepackage{multirow}

\usepackage{float}

\usepackage{wasysym}

\usepackage[T1]{fontenc}
\usepackage{titlesec}

\setlength{\parindent}{3ex}

% Литература
\addto\captionsrussian{\def\refname{Литература}}

%Заговолок
\title{
	\center{\textbf{Контрольные вопросы}}
	}


\begin{document}	% the beginning of the document

\maketitle

\section{Перечислите все специальные функции-члены класса, включая перемещающие операции.}
	
Специальные функции-члены класса:

	\begin{itemize}
	 
	\item конструктор по умолчанию;

	\item пользовательские конструкторы;
	
	\item деструктор;
	
	\item copy assignment оператор;
	
	\item copy конструктор;
	
	\item move assignment оператор;
	
	\item move конструктор.
	
	\end{itemize}

\newpage

\section{Приведите примеры операторов, которые можно, нельзя и не рекомендуется перегружать.}

	Нельзя перегрузить операторы:
	
	\begin{itemize}
	
	\item conditional (?:);
	\item sizeof;
	\item scope (::);
	\item member selector (.);
	\item member pointer selector (.*);
	\item typeid;
	\item casting operators.
	
	\end{itemize}

	Все остальные операторы могут быть перегружены, например:
	
	\begin{itemize}
	
	\item plus operator (+);
	\item binary minus operator (-);
	\item multiplication operator (*);
	\item division operator (/);
	\item многие другие...
	
	\end{itemize}

	Наконец, не рекомендуется перегружать операторы:
	
	\begin{itemize}
	
	\item logical 'and' operator (\&);
	
	\item logical 'or' operator (||);
	
	\item comma operator (,).
	
	\end{itemize}
	
\newpage

\section{О каких преобразованиях стоит помнить при проектировании операторов?}
	
	При проектировании операторов следует помнить о:
		
	\begin{itemize}
	
	\item перегруженном приведении типов (с помощью перегрузки оператора приведения типов); 
	
	\item о неявном приведении типов (с помощью перегруженных пользовательских конструкторов, поэтому для этого необходимо наличие конструктора без ключевого слова explicit).
	
	\end{itemize}

	Это связано с тем, что компилятор работает по следующему принципу: если какой-либо из операндов оператора является пользовательским типом данных, компилятор проверяет, есть ли у данного типа соответствующая перегруженная операторная функция, которую он может вызвать, но, если он не может найти её, он попытается преобразовать один или несколько операндов пользовательского типа в фундаментальные типы данных, чтобы он мог использовать соответствующий встроенный оператор (через перегруженное приведение типов). Если это не удается, то это приведет к ошибке компиляции. На самом деле, может возникнуть неоднозначность выбора между неявным преобразованием с помощью конструтора одного операнда, не принадлежащего пользовательскому типу, и перегруженным приведением другого операнда, являющегося объектом пользовательского типа. Это приводит к ошибке ambiguous overload.

\newpage

\section{Опишите классификацию выражений на основе перемещаемости и идентифицируемости.}

	Идентифицируемость (identity) -- свойство выражения, заключающееся в наличии какого-либо параметра, по которому можно понять, ссылаются ли два выражения на одну и ту же сущность или нет. Перемещаемость (mobility) -- свойство выражения, заключающееся в возможности поддерживания семантики перемещения.
	
	Обладающие идентичностью выражения обобщены под термином glvalue (generalized values), перемещаемые выражения называются rvalue. Комбинации двух этих свойств определили 3 основные категории выражений:
	
	\begin{itemize}
	
	\item lvalue -- идентифицируемое неперемещаемое выражение;
	
	\item xvalue -- идентифицируемое перемещаемое выражение;
	
	\item prvalue -- неидентифицируемое перемещаемое выражение.	
	
	\end{itemize}

	На самом деле, в Стандарте C++17 появилось понятие избегание копирования (copy elision) -- формализация ситуаций, когда компилятор может и должен избегать копирования и перемещения объектов. В связи с этим, prvalue не обязательно могут быть перемещены.

\newpage

\section{Зачем нужны rvalue ссылки?}

	rvalue ссылки позволяют:
	
	\begin{itemize}
	
	\item продлить срок жизни объекта, которым они инициализированы до срока жизни rvalue ссылки (подобно lvalue ссылкам на константные объекты);
	
	\item неконстантные rvalue ссылки позволяют изменять rvalue.
	
	\end{itemize}		
	
	Правда, эти свойства используются редко. Гораздо чаще rvalue ссылки используются в качестве аргументов функции. Это бывает особенно полезно, при перегрузки функций, когда имеет смысл разграничить реализации функций для lvalue и rvalue ссылок. Это свойство используется для реализации семантики перемещения. Поэтому rvalue ссылки дают возможность сделать код семантически более правильным, привнести в него дополнительную скорость за счёт семантики перемещения, а также предоставляют надежное средство для передачи параметров во внутренние функции без большого количества перегруженных функций.

\newpage

\section{Почему семантика перемещения лучше копирования?}

	 Семантика перемещения позволяет повторно использовать объекты, время жизни которых приближается к концу, с помощью операций перемещения, что позволяет в некоторых ситуациях избежать дорогостоящих операций копирования (особенно "глубокого" копирования).  Благодаря операциям перемещения вместе с самим объектом могут быть перемещены связанные с ним сущности, что может быть более удобно (и даже более безопасно) чем их копирование. Кроме того, с помощью семантики перемещений можно реализовать объекты, которые не могут копироваться (или их копирование нежелательно).

	Но не всё однозначно. Так, С. Мейерс пишет: "Перемещающие операции не всеrда дешевле копирования, а когда и дешевле, то не всегда настолько, как вы думаете; кроме того, они не всегда вызываются в контексте, где перемещение является корректным. Конструкция type \&\& не всегда представляет rvalue-ccылкy".	 

\newpage

\section{Что делает функция std::move и когда нет необходимости явно её вызывать?}

	Функция std::move выполняет безусловное приведение своего аргумента к rvalue. Функция std::move ничего не перемещает (более того она даже не
гарантирует, что приведенный этой функцией объект будет иметь право быть перемещенным).
	
	Функцию std::move нет необходимости явно вызывать, если она:
	
	\begin{itemize}
	
	\item предотвращает выполнение компилятором оптимизации copy/move elision;
	
	\item избыточна, поскольку уже используется неявно.	
	
	\end{itemize}
	
	Первый случай реализуется в результате оптимизации именованного возвращаемого значения (NRVO). Это возможно, если, например, возвращаемая функцией переменная является non-volatile (значение не может меняться извне и компилятор не будет оптимизировать эту переменную) локальным объектом некоторого класса и не является параметром функции. Тогда компилятор может создать объект непосредственно в месте его конечного назначения (ячейке стека вызова). Использование std::move в данном случае будет конфликтовать с реализацией NRVO, требующего возврата имени в функции, в то время как std::move возвращает ссылку.
	
	Второй случай реализуется в результате оптимизации возвращаемого значения (RVO) при некоторых условиях, когда невозможно выполнение оптимизации copy/move elision компилятором. Это возможно, если, например, возвращаемая функцией переменная является объектом некоторого класса и при этом -- ещё и аргументом данной функции. В таком случае C++ гарантирует использование операции перемещения по умолчанию и выполнение two-stage overload resolution как будто объект являлся rvalue. Аргумент функции, будучи lvalue, становится xvalue, так как он находится "на грани уничтожения" (он действительно исчезнет после выхода из области видимости функции). В результате при возвращении функцией значения неявно используется std::move, и она возвращает rvalue-объект.
	
\newpage

\section{Кем выполняется непосредственная работа по перемещению?}

	Непосредственная работа по перемещению выполняется специальными перемещающими операциями, имплементируемыми специальными функциями-членами класса: move assignment оператором (=) и move конструктором. Первый работает непосредственно с объектами класса, а второй служит для инициализации.
	
	Move конструктор и move assignment оператор вызываются, когда эти функции определены в классе, а аргументом для конструктора или присваивания является rvalue, часто являющееся литералом или временным значением.

	В большинстве случаев move конструктор перемещения и move assignment оператор не создаются по умолчанию, если в классе нет чего-либо из следующих специальных функций-членов: copy конструктора, copy assignment оператора, move assignment оператора или деструктора.

\newpage

\section{Когда может потребоваться пользовательская реализация специальных функций-членов класса?}

	Использование shallow (memberwise) copy (поверхностные копии), реализуемое с помощью copy конструктора по умолчанию copy assignment оператора по умолчаниюы, может привести к UB. Так, shallow copy указателя может привести к проблеме, когда объект класса и его копия имеют разные указатели, ссылающиеся на одну и ту же сущность, что потенциально может привести к висячему указателю. От этой проблемы можно избавиться с помощью deep copy (глубокого копирования), благодаря которому разные указатели будут ссылаться на разные копии. Тогда и необходима пользовательская реализация специальных функций-членов. В целом, стоит помнить что:
	
	\begin{itemize}
	
	\item copy конструктор по умолчанию и copy assignment операторы по умолчанию делают shallow copy, что хорошо для классов, не содержащих динамически выделяемых переменных;
	
	\item классы с динамически выделяемыми переменными должны иметь copy конструктор и copy assignment оператор присваивания, которые выполняют deep copy;
	
	\item классы из стандартной библиотеки более предпочтительны, чем собственное управление памятью (там многое уже реализовано).

	\end{itemize}

\newpage

\section{Для чего нужны ключевые слова default и delete в объявлении специальных функций-членов класса?}

	Каждый класс может явно выбирать, какие из специальных функций-членов существуют с их определением по умолчанию, а какие удаляются с помощью ключевых слов default и delete соответственно.

	Ключевое слово default в объявлении специальных функций членов-классов необходимо для создания специальной функции-члена (default/move/copy) по умолчанию без аргументов. При этом соответствующие пользовательские конструкторы не должны иметь все аргументы по умолчанию (это касается default конструктора). Кроме того, default повышает читабельность кода.
	
	Ключевое слово delete в объявлении специальных функций членов-классов необходимо для запрета на использование данной специальной функции-члена компилятором. Любое использование deleted функции (например, при явном преобразовании типов) приведёт к ошибке компиляции.

\newpage

\addcontentsline{toc}{section}{Литература}
 
\begin{thebibliography}{}
    \bibitem{litlink1} https://www.learncpp.com/
    \bibitem{litlink2} https:http://scrutator.me/post/2011/08/02/rvalue-refs.aspx
    \bibitem{litlink3} https://habr.com/ru/post/441742/
	\bibitem{litlink4} Мейерс, Скотт. Эффективный и современный С++: 42 рекомендации по исполыованию С++11 и С++14.: Пер. с англ. -- М. : ООО "ИЛ. Вильямс", 2016. -- 304 с.: ил. -- Пapал. тит. англ.
	\bibitem{litlink5} https://cplusplus.com/
	\bibitem{litlink6} https://stackoverflow.com/questions/60447951/ambiguity-with-conversion-operator-and-constructorы
    
\end{thebibliography}


\end{document} % end of the document
